\documentclass[12pt]{article}

% Packages
\usepackage[margin=1in]{geometry}
\usepackage{times}
\usepackage{setspace}
\usepackage{booktabs}
\usepackage{longtable}
\usepackage{array}
\usepackage{hyperref}
\usepackage{enumitem}
\usepackage{amssymb}
\usepackage[style=apa,backend=biber]{biblatex}

% Bibliography configuration
\addbibresource{references.bib}

% Document settings
\doublespacing
\setlength{\parindent}{0.5in}

% Title and author
\title{Predicting NIMBYism}
\author{Daniel Lewis\\
dl3645@columbia.edu\\
Columbia University\\
Urban Planning}
\date{December 10, 2025}

\begin{document}

\maketitle

\section{Introduction}

This thesis investigates the patterns of neighborhood opposition to housing development in Austin, Texas, and evaluates the democratic implications of using predictive analytics to anticipate such behavior. The project asks: how can cities identify and understand patterns of neighborhood opposition, and what are the normative consequences of using algorithmic tools to manage this participation?

Austin, Texas faces an unprecedented housing affordability crisis that has fundamentally reshaped its political landscape. Between 2010 and 2020, Austin's population grew by 33\%, adding nearly 250,000 residents \autocite{census2020}. This growth, driven primarily by the technology sector, created severe supply constraints. By May 2022, the median home price reached \$550,000, up 79\% from January 2020. As of August 2025, prices have moderated to \$444,490 but remain unaffordable for most residents \autocite{unlockMLS2025}. The consequences extend beyond housing costs: displacement of long-term residents, lengthening commutes, and growing tension between neighborhood preservation and regional growth have become defining features of Austin's political discourse.

The research addresses a critical gap at the intersection of urban planning and algorithmic governance. As cities turn to technological solutions to streamline development, the use of predictive tools raises questions about democratic participation and civil liberties. The findings will establish ethical frameworks for cities considering such approaches and provide empirical evidence about the tradeoffs between efficiency and legitimacy in urban governance. In Austin specifically, the Development Services Department could use these insights to improve community engagement strategies.

\textbf{Research Questions}

The project is guided by three questions:
\begin{enumerate}
\item[a.] What demographic, geographic, and behavioral factors predict whether residents will oppose specific development projects?
\item[b.] How do different stakeholders perceive the legitimacy and utility of algorithmic tools in housing policy?
\item[c.] What alternative technological approaches could improve development processes without raising concerns about democratic participation?
\end{enumerate}

An important recent development shapes this work. In 2025, the Texas Legislature passed House Bill 24, raising the valid petition protest threshold from 20\% to 60\% and eliminating the automatic supermajority requirement (effective Sept 1, 2025). While these changes occurred after our validation period (we use data from 2007 onward but test and validate models on 2018-2025), they create crucial context. Even though the petition mechanism in Texas has been substantially weakened, we may still recover predictable patterns of opposition applicable both in Texas and in the roughly 20 other states that continue to operate under similar rules.

\section{Background}

Austin's governance structure complicates housing policy implementation. The city operates under a council-manager system where the mayor has limited executive authority. Passing controversial reforms has historically required council supermajorities to overcome property owner protests authorized by Texas law \autocite{cityaustin2010}. Under the pre-HB 24 regime, property owners within 200 feet of a proposed rezoning could trigger a ``valid petition'' requiring a three-fourths supermajority for approval if 20\% of affected landowners signed.

Historically, neighborhood associations have mounted organized opposition to development using tools like valid petitions and public testimony. This opposition often correlated with neighborhood demographics, particularly homeownership rates, property values, and racial composition, raising questions about equity in whose voices shape housing outcomes. In response to the deepening crisis, Austin underwent a political transformation. The November 2022 elections delivered a pro-housing supermajority \autocite{tcclerk2022}. This shift enabled sweeping reforms: eliminating parking requirements citywide in November 2023 \autocite{axios2023parking}, reducing minimum lot sizes, and allowing density on corner lots through the HOME Initiative \autocite{pool2023}.

These reforms have generated measurable impacts, with apartment vacancy rates climbing to 10-15\% by 2025 \autocite{kvue2024vacancy}. However, implementation challenges persist, with permit volumes fluctuating and staffing pressures in the Development Services Department. Throughout this period, opposition tactics have evolved from formal protest petitions toward public testimony and organized attendance at hearings. Austin's unique position as a technology hub, home to major tech firms and a workforce of approximately 180,000, provides a distinct civic context for debating the role of data-driven tools in managing political conflict. The city has been nationally recognized for innovative civic technology initiatives, making stakeholders particularly well-positioned to engage substantively with questions about algorithmic tools in planning \autocite{efa2025,idc2024}.

This research contributes to literature on algorithmic governance and housing policy by separating the empirical structure of opposition from the normative question of prediction, providing evidence on the tradeoffs between efficiency and legitimacy that informs theoretical debates about technocratic city management.


\section{Literature Review}

This section organizes the literature around three themes corresponding to the research questions.

\subsection{Theme A: Homeownership and Opposition to Housing}

William Fischel's ``Homevoter Hypothesis'' \autocite{fischel2001} argues that homeowners rationally oppose development that might reduce property values, creating systematic opposition to housing supply increases. McCabe shows that homeowner-dominated municipalities are more likely to restrict multifamily housing \autocite{mccabe2016}. Hankinson finds that even renters exhibit NIMBY attitudes when experiencing ``price anxiety'' about future housing costs \autocite{hankinson2018}. We will test whether these variables predict opposition behavior in Austin.

\subsection{Theme B: Machine Learning and Democratic Legitimacy}

Kleinberg et al.\ show that ML models can improve prediction of judicial decisions while raising concerns about delegating high-stakes judgments to algorithms \autocite{kleinberg2018}. Eubanks documents how predictive systems intensify surveillance of marginalized populations \autocite{eubanks2018}. Barocas and Selbst show how seemingly neutral variables can function as proxies for protected attributes \autocite{barocas2016}. Our study constructs a parcel-level predictive model of zoning opposition while pairing that work with qualitative interviews about legitimacy and fairness.

\subsection{Theme C: Limits of Algorithmic Governance}

Kitchin argues that data-driven systems prioritize technocratic efficiency over democratic accountability \autocite{kitchin2014}. Schuilenburg and Peeters show how algorithmic systems subtly reshape who participates in public life \autocite{schuilenburg2018}. Zarsky argues that opaque decisions undermine individuals' ability to contest outcomes \autocite{zarsky2016}. Our qualitative interviews will examine whether stakeholders view NIMBY prediction as legitimate or as surveillance that chills participation.


\section{Research Design}

This study combines quantitative analysis of existing administrative records with qualitative interviews and public meeting observations.

\subsection{Research Site Selection}

Austin, Texas is unique among major U.S.\ metropolitan areas in maintaining comprehensive public records of protest petitions and formal opposition to zoning changes. A public information request survey of the top 50 major metros revealed that Austin alone has preserved these records in a form amenable to parcel-level analysis. The study period (data from 2007 onward, validation on 2018-2025) spans both zoning stasis and dramatic reform, including HOME Phase 1 (December 2023) and HOME Phase 2 (May 2024). Austin's civic tech infrastructure \autocite{efa2025,idc2024} and diverse stakeholder landscape provide rich context.

\subsection{Quantitative Component}

\textbf{Data Sources.} The quantitative analysis will link: opposition records (2007-2025) from zoning case files; City of Austin development records from the Open Data Portal; Travis Central Appraisal District property records \autocite{tcad_public}; Census tract-level demographics; and public testimony from City Council meetings.

\textbf{Variables.} Predictors include property characteristics (assessed value, homeownership, tenure), spatial variables (proximity to developments, neighborhood demographics), and behavioral variables (prior testimony, policy regime indicators).

\textbf{Modeling.} We will compare logistic regression and tree-based methods (Random Forests, Gradient Boosted Trees). We use an annual expanding window validation strategy training on pre-2018 data and validating on 2018-2025 to avoid data leakage and mimic real-world deployment.

\subsection{Qualitative Component}

Approximately 10 semi-structured interviews (15-30 minutes) with stakeholders: city officials, neighborhood leaders, housing advocates, and civic technologists. We will use purposive sampling and do not claim thematic saturation. Non-participant observation of public meetings will supplement interviews. Analysis will follow an iterative thematic coding approach using deductive codes adapted from legitimacy and procedural justice frameworks.

\subsection{Integration}

The mixed-methods design triangulates empirical patterns (quantitative) with normative stakeholder evaluations (qualitative) to address the research questions.



\section{Task Schedule}
{\small

\noindent\textbf{I.\quad 2025 October--November: Literature Review and Data Acquisition}
\begin{itemize}[noitemsep,leftmargin=2em]
    \item 10/01--11/15: Complete comprehensive literature review
    \item 11/25: Submit IRB materials (target approval mid-December)
    \item 10/15-12/15: Acquire datasets via public information requests
    \item \textbf{Milestone}: Complete thesis proposal (12/10)
\end{itemize}

\noindent\textbf{II.\quad 2025 December -- 2026 January: Quantitative Analysis}
\begin{itemize}[noitemsep,leftmargin=2em]
    \item 12/15--01/15: Data preparation, cleaning, and linkage; construct two-table structure
    \item 01/10--01/25: Feature construction and exploratory analysis
    \item 01/15--02/10: ML model development with annual expanding window validation
    \item 01/20--02/10: Geographic and temporal analysis of opposition patterns
    \item \textbf{Deliverable}: Draft quantitative methods and results chapter (02/10)
\end{itemize}

\noindent\textbf{III.\quad 2026 January--February: Qualitative Data Collection}
\begin{itemize}[noitemsep,leftmargin=2em]
    \item 01/05--01/25: Finalize interview protocol, recruit participants via purposive sampling
    \item 01/20--02/28: Conduct 10 interviews (15--30 minutes each)
    \item 01/15--03/15: Non-participant observation (live attendance and archived ATXN review)
    \item 02/01--03/10: Transcribe and begin preliminary coding
    \item \textbf{Deliverable}: Complete interview transcripts and field notes (03/10)
    \item \textbf{Note}: Schedule in-person Austin time during this period, secure Columbia travel funding
\end{itemize}

\noindent\textbf{IV.\quad 2026 February--March: Integration and Analysis}
\begin{itemize}[noitemsep,leftmargin=2em]
    \item 02/20-03/20: Systematic qualitative coding
    \item 03/01--03/15: Analyze comparative frameworks from other cities
    \item 03/10--03/25: Integrate quantitative and qualitative findings
    \item 03/15--03/30: Develop theoretical framework connecting findings to HB 24 context
    \item \textbf{Deliverable}: Complete draft chapters submitted to advisor (03/30)
\end{itemize}

\noindent\textbf{V.\quad 2026 March--May: Writing and Defense}
\begin{itemize}[noitemsep,leftmargin=2em]
    \item 04/01--04/10: Revise based on advisor feedback
    \item 04/03: Distribute penultimate draft to reader(s) (14 days before jury, required)
    \item 04/05--04/15: Prepare jury presentation
    \item 04/13--04/17: Thesis jury (required for graduation)
    \item 04/18--05/05: Final revisions incorporating jury feedback
    \item \textbf{Final submission}: 05/08/2026
\end{itemize}
}

% Print bibliography
\printbibliography[title=References]

\appendix

\section{Interview Guide}

\textbf{Predicting NIMBYism: Stakeholder Interview Guide}

\textbf{Introduction}
\begin{itemize}[leftmargin=2em]
    \item Thank you for agreeing to speak with me. I am a graduate student at Columbia University studying housing development processes in Austin.
    \item This interview will last approximately 15-30 minutes.
    \item Participation is voluntary. You may decline to answer any question or stop at any time.
\end{itemize}

\textit{If recording}: Do I have your permission to record this conversation for accuracy?

\textbf{I. Professional Background and Context}
\begin{enumerate}[leftmargin=2em]
    \item Could you briefly describe your role and how you interact with housing development or zoning processes in Austin?
    \item Based on your experience, what are the most significant factors that currently delay or prevent housing development in Austin?
    \item How would you characterize the nature of community engagement or neighborhood opposition to revisions/re-zonings in recent years?
\end{enumerate}

\textbf{II. Perceptions of Predictive Tools}
\begin{enumerate}
    \item Are you aware of any data-driven or algorithmic tools currently used by the City of Austin (e.g., in permitting, service delivery, or planning) that you find particularly useful or problematic?
    \item Cities are increasingly using data to model urban trends. What is your initial reaction to the concept of using data to predict where neighborhood opposition to zoning changes is most likely to occur?
    \item What potential benefits, if any, could you see from using such a tool? (For example: better resource allocation, earlier outreach, identifying engagement gaps).
    \item What risks or concerns would you have about such a tool? (For example: bias, transparency, fairness, exclusion).
\end{enumerate}

\textbf{III. Democratic Legitimacy and Governance}
\begin{enumerate}
    \item Do you believe that using predictive analytics to anticipate public participation would make the planning process more or less democratic? Why?
    \item If the City were to adopt more predictive technologies for planning, what specific safeguards or transparency measures would be essential for you to trust them?
    \item Is there anything else about the intersection of technology, data, and housing policy in Austin that you would like to share?
\end{enumerate}

\section{Recruitment Materials}

\textbf{Email Subject:} Interview Request: Research on Austin Housing Policy and Technology

Dear [Name],

I am writing to invite you to participate in a research study regarding housing development processes in Austin. I am a graduate student in Urban Planning at Columbia University, conducting this research for a thesis under the supervision of Dr. Hiba Bou Akar.

\textbf{Purpose of the Study:}
This study examines patterns of neighborhood opposition to housing development and evaluates how different stakeholders perceive the use of new data-driven and predictive tools in planning. We are seeking perspectives from [City Officials / Neighborhood Leaders / Housing Advocates / Developers / Civic Technologists] to understand diverse views on transparency, efficiency, and fairness in the development process.

\textbf{Procedures:}
Participation involves a single semi-structured interview lasting approximately 15--30 minutes. The interview can be conducted in person, by phone, or via secure videoconference (e.g., Zoom).

\textbf{Voluntary Participation:}
Your participation is completely voluntary. You may decline to answer any questions or withdraw from the study at any time without penalty.

If you are willing to share your perspective, please let me know your availability for a brief conversation in the coming weeks.

Thank you for your time and consideration.

Sincerely,

\noindent Daniel Lewis, Investigator\\\noindent Graduate School of Architecture, Planning and Preservation\\\noindent Columbia University\\\noindent dl3645@columbia.edu

\section{Public Meeting Observation Protocol}

\textbf{Date}: \_\_\_\_\_\_\_\_\_\_\_\_\_ \\
\textbf{Location}: \_\_\_\_\_\_\_\_\_\_\_\_\_ \\
\textbf{Meeting Type}: $\Box$ City Council \hspace{0.5cm} $\Box$ Planning \& Zoning Commission \hspace{0.5cm} $\Box$ Neighborhood Association \\
\textbf{Format}: $\Box$ Live attendance \hspace{0.5cm} $\Box$ Archived ATXN video review

\textbf{Attendees (general categories, no names)}:
\begin{itemize}
    \item Approximate number present: \_\_\_\_\_\_\_
    \item Stakeholder groups represented: \_\_\_\_\_\_\_
\end{itemize}

\textbf{Topics Discussed Related to Study}:
\begin{itemize}
    \item Housing development or zoning matters: \_\_\_\_\_\_\_
    \item References to data, predictive tools, or algorithmic approaches: \_\_\_\_\_\_\_
    \item Discussion of neighborhood opposition or participation: \_\_\_\_\_\_\_
\end{itemize}

\textbf{Field Notes}:
[Space for detailed notes on how stakeholders discuss housing development processes, predictive tools, democratic participation, and related themes. Focus on publicly observable statements and behavior that are already part of the public record.]

\textbf{Follow-up Questions or Observations}:

\end{document}
