\documentclass[12pt]{article}

% Packages
\usepackage[margin=1in]{geometry}
\usepackage{times}
\usepackage{setspace}
\usepackage{booktabs}
\usepackage{longtable}
\usepackage{array}
\usepackage{hyperref}
\usepackage{enumitem}

% Document settings
\doublespacing
\setlength{\parindent}{0.5in}

% Title and author
\title{Assignment 2: Understanding NIMBY Opposition Patterns in Austin's Housing Reform Era}
\author{Daniel Lewis\\
dl3645@columbia.edu\\
Columbia University\\
Urban Planning}
\date{October 7, 2025}

\begin{document}

\maketitle

\section{Introduction: Austin's Housing Crisis and Political Transformation}

Austin, Texas faces an unprecedented housing affordability crisis that has fundamentally reshaped its political landscape. Between 2010 and 2020, Austin's population grew by 33\%, adding nearly 250,000 residents to reach 978,908 (U.S. Census Bureau, 2020). This growth, driven primarily by the technology sector expansion, created severe housing supply constraints. By May 2022, the median home price reached \$550,000, up 79\% from January 2020's \$308,000. As of August 2025, prices have moderated to \$444,490 but remain unaffordable for most residents (UnlockMLS, 2025).

The crisis has displaced long-term residents, particularly communities of color. Austin's Black population declined by approximately 5,000 residents from 2010-2020, with the city's Black population share falling to 7.8\%. The Hispanic population, while growing modestly in absolute numbers, saw its share of the city population decline from 35.1\% to 32.5\% as gentrification pushed residents to suburban areas (Census, 2020). An Austin Fire Department survey found that 70\% of emergency medical services members lived outside city limits due to housing costs, with average one-way commutes of 45 minutes.

Austin's governance structure complicates housing policy implementation. The city operates under a council-manager system with 10 single-member districts plus an at-large mayor. The mayor has minimal executive authority---one vote equal to other council members, no veto power, and no ability to hire department heads. All executive functions rest with the City Manager, who reports to all 11 council members equally. This structure means building council supermajorities is essential for passing controversial reforms, as Texas law requires three-quarters votes to overcome property owner protests against zoning changes.

In response to the crisis, Austin underwent a dramatic political transformation. The November 2022 elections delivered a pro-housing supermajority, with candidates supporting increased housing production winning all five contested seats. Most dramatically, Council Member Leslie Pool, who had campaigned on ``Protect Single-Family Zoning'' platforms, reversed her position in July 2023 to author the HOME (Housing Opportunities for Middle-Income Earners) Initiative. This political shift enabled sweeping reforms: eliminating parking requirements citywide (November 2023), reducing minimum lot sizes from 5,750 to 1,800 square feet, and allowing up to 10 units on corner lots.

The reforms have generated measurable market impacts. Austin's apartment vacancy rate climbed from 3.96\% in September 2021 to approximately 10-15\% by mid-2025. Housing permits increased 86\% from 15,000 in 2023 to 28,000 in 2024. However, implementation faces significant challenges. The Development Services Department cut 24 positions in FY 2024 despite increased application volume. Site plan review times, while improved from 87 to 32 business days initially, face new delays from the surge in complex applications.

Throughout this transformation, neighborhood associations have mounted organized opposition to development projects, using tools like valid petitions, public testimony, and legal challenges. Understanding the patterns and predictors of this opposition has become crucial for implementing housing reforms effectively. Austin's unique position as a technology hub---home to Dell, Oracle, Tesla, Apple, and 180,000 tech workers---provides both the technical capability and the civic context to examine whether data-driven approaches could improve housing policy implementation.

\section{Research Questions}

This project aims to answer the following primary question:

\textbf{How can cities identify and understand patterns of neighborhood opposition to housing development, and what are the democratic implications of using predictive analytics to anticipate such opposition?}

Sub-questions:

\begin{enumerate}
\item[a.] What demographic, geographic, and behavioral factors predict whether residents will oppose specific development projects in Austin?

\item[b.] How do different stakeholder groups (homeowners, renters, housing advocates, city officials) perceive the legitimacy and utility of algorithmic tools in housing policy implementation?

\item[c.] What alternative technological approaches could improve housing development processes without raising concerns about democratic participation?
\end{enumerate}

\section{Describe the ``so what''}

This research addresses a critical gap at the intersection of urban planning, democratic governance, and algorithmic decision-making. As cities nationwide struggle with housing affordability, many are turning to technological solutions to streamline development approval. However, the use of predictive algorithms to identify potential opposition raises fundamental questions about democratic participation and civil liberties that have not been systematically examined.

The findings will directly inform policy decisions in Austin and other cities considering similar tools. Austin's Development Services Department and Planning Commission could use insights about opposition patterns to improve community engagement and resource allocation. Housing advocates could better understand how to build coalitions and address legitimate concerns. Most importantly, the research will establish ethical frameworks for cities considering algorithmic tools in democratic processes.

This work contributes to broader literature on algorithmic governance, smart cities, and housing policy. It provides empirical evidence about the tradeoffs between efficiency and legitimacy in urban governance, informing theoretical debates about technocratic versus democratic approaches to city management.

\section{Research Design Grid}

\begin{longtable}{p{4cm}|p{5.5cm}|p{5.5cm}}
\toprule
\textbf{Research Question} & \textbf{Data Sources} & \textbf{Details/Analysis} \\
\midrule
\endfirsthead

\multicolumn{3}{c}{\textit{(continued from previous page)}} \\
\toprule
\textbf{Research Question} & \textbf{Data Sources} & \textbf{Details/Analysis} \\
\midrule
\endhead

\midrule
\multicolumn{3}{r}{\textit{(continued on next page)}} \\
\endfoot

\bottomrule
\endlastfoot

What factors predict opposition to development projects? &
\begin{itemize}[noitemsep,topsep=0pt,parsep=0pt,partopsep=0pt,leftmargin=*]
\item Travis Central Appraisal District property tax protest records (2018-2025)
\item City Council meeting testimony transcripts (10 years)
\item Campaign contribution records (Texas Ethics Commission)
\item Census demographic data by tract
\item Development application data (location, type, size)
\end{itemize} &
\begin{itemize}[noitemsep,topsep=0pt,parsep=0pt,partopsep=0pt,leftmargin=*]
\item Machine learning model to identify predictive factors
\item Variables: proximity to development, home value, length of residence, age, previous testimony
\item Validation on 2018-2022 cases, testing on 2023-2025
\item Geographic analysis of opposition clustering
\item Temporal analysis of opposition evolution
\end{itemize} \\

\midrule

How do stakeholders perceive algorithmic prediction tools? &
\begin{itemize}[noitemsep,topsep=0pt,parsep=0pt,partopsep=0pt,leftmargin=*]
\item Semi-structured interviews (n=40):
\begin{itemize}[noitemsep,topsep=0pt]
\item 10 city officials/planners
\item 10 neighborhood association leaders
\item 10 housing advocates (AURA members)
\item 10 developers
\end{itemize}
\item Public meeting observations (6 months)
\item Document analysis of public comments
\end{itemize} &
\begin{itemize}[noitemsep,topsep=0pt,parsep=0pt,partopsep=0pt,leftmargin=*]
\item Thematic analysis of interview transcripts
\item Stakeholder mapping of concerns and benefits
\item Comparison of perceived vs actual algorithmic capabilities
\item Analysis of democratic legitimacy concerns
\item Trust and transparency themes
\end{itemize} \\

\midrule

What alternative technological approaches exist? &
\begin{itemize}[noitemsep,topsep=0pt,parsep=0pt,partopsep=0pt,leftmargin=*]
\item Comparative case studies:
\begin{itemize}[noitemsep,topsep=0pt]
\item Seattle traffic signal optimization (Project Green Light with Google AI)
\item Boston Residential Displacement Risk Map (city risk-scoring tool)
\item San Francisco sentiment mapping research (Twitter/311 analysis prototypes)
\end{itemize}
\item Austin's deployed tools:
\begin{itemize}[noitemsep,topsep=0pt]
\item Build + Connect Portal metrics
\item Development Calculator usage data
\item Housing Dashboard analytics
\end{itemize}
\item Expert interviews (n=15) with civic technologists
\end{itemize} &
\begin{itemize}[noitemsep,topsep=0pt,parsep=0pt,partopsep=0pt,leftmargin=*]
\item Comparative analysis of different cities' approaches
\item Effectiveness metrics (processing time, public satisfaction)
\item Democratic impact assessment
\item Cost-benefit analysis
\item Best practices identification
\item Framework development for ethical deployment
\end{itemize} \\

\end{longtable}

\end{document}