\documentclass[12pt]{article}

% Packages
\usepackage[margin=1in]{geometry}
\usepackage{times}
\usepackage{setspace}
\usepackage{booktabs}
\usepackage{longtable}
\usepackage{array}
\usepackage{hyperref}
\usepackage{enumitem}
\usepackage{amssymb}
\usepackage{natbib}
\bibliographystyle{apalike}

% Document settings
\doublespacing
\setlength{\parindent}{0.5in}
\widowpenalty=10000
\clubpenalty=10000

% Title and author
\title{Predicting NIMBYism}
\author{Daniel Lewis\\
dl3645@columbia.edu\\
Columbia University\\
Urban Planning}
\date{December 10, 2025}

\begin{document}

\maketitle

\section{Introduction}

This thesis investigates the patterns of neighborhood opposition to housing development in Austin, Texas, and evaluates the democratic implications of using predictive analytics to anticipate such behavior. The project asks: how can cities identify and understand patterns of neighborhood opposition, and what are the normative consequences of using algorithmic tools to manage this participation?

Austin, Texas faces an unprecedented housing affordability crisis that has fundamentally reshaped its political landscape. Between 2010 and 2020, Austin's population grew by 33\%, adding nearly 250,000 residents \autocite{census2020}. This growth, driven primarily by the technology sector, created severe supply constraints. By May 2022, the median home price reached \$550,000, up 79\% from January 2020. As of August 2025, prices have moderated to \$444,490 but remain unaffordable for most residents \autocite{unlockMLS2025}. The consequences extend beyond housing costs: displacement of long-term residents, lengthening commutes, and growing tension between neighborhood preservation and regional growth have become defining features of Austin's political discourse.

The research addresses a critical gap at the intersection of urban planning and algorithmic governance. As cities turn to technological solutions to streamline development, the use of predictive tools raises questions about democratic participation and civil liberties. The findings will establish ethical frameworks for cities considering such approaches and provide empirical evidence about the tradeoffs between efficiency and legitimacy in urban governance. In Austin specifically, the Development Services Department could use these insights to improve community engagement strategies.

\textbf{Research Questions}

The project is guided by three questions:
\begin{enumerate}
\item[a.] What demographic, geographic, and behavioral factors predict whether residents will oppose specific development projects?
\item[b.] How do different stakeholders perceive the legitimacy and utility of algorithmic tools in housing policy?
\item[c.] What alternative technological approaches could improve development processes without raising concerns about democratic participation?
\end{enumerate}

An important recent development shapes this work. In 2025, the Texas Legislature passed House Bill 24, raising the valid petition protest threshold from 20\% to 60\% and eliminating the automatic supermajority requirement (effective Sept 1, 2025). Even though the petition mechanism in Texas has been substantially weakened, we may still recover predictable patterns of opposition applicable anywhere in the United States, including the roughly 20 other states that continue to operate under similar formal protest mechanisms.

\section{Background}

Austin's governance structure complicates housing policy implementation. The city operates under a council-manager system where the mayor has limited executive authority. Passing controversial reforms has historically required council supermajorities to overcome property owner protests authorized by Texas law \autocite{cityaustin2010}. Under the pre-HB 24 regime, property owners within 200 feet of a proposed rezoning could trigger a ``valid petition'' requiring a three-fourths supermajority for approval if 20\% of affected landowners signed.

Historically, neighborhood associations have mounted organized opposition to development using tools like valid petitions and public testimony. This opposition often correlated with neighborhood demographics, particularly homeownership rates, property values, and racial composition, raising questions about equity in whose voices shape housing outcomes. In response to the deepening crisis, Austin underwent a political transformation. The November 2022 elections delivered a pro-housing supermajority \autocite{tcclerk2022}. This shift enabled sweeping reforms: eliminating parking requirements citywide in November 2023 \autocite{axios2023parking}, reducing minimum lot sizes, and allowing density on corner lots through the HOME Initiative \autocite{pool2023}.

These reforms have generated measurable impacts, with apartment vacancy rates climbing to 10-15\% by 2025 \autocite{kvue2024vacancy}. However, implementation challenges persist, with permit volumes fluctuating and staffing pressures in the Development Services Department. Throughout this period, opposition tactics have evolved from formal protest petitions toward public testimony and organized attendance at hearings. Austin's unique position as a technology hub, home to major tech firms and a workforce of approximately 180,000, provides a distinct civic context for debating the role of data-driven tools in managing political conflict. The city has been nationally recognized for innovative civic technology initiatives, making stakeholders particularly well-positioned to engage substantively with questions about algorithmic tools in planning \autocite{efa2025,idc2024}.

This research contributes to literature on algorithmic governance and housing policy by separating the empirical structure of opposition from the normative question of prediction, providing evidence on the tradeoffs between efficiency and legitimacy that informs theoretical debates about technocratic city management.


\section{Literature Review}

This section organizes the literature around three themes that correspond to the three research questions introduced above. Theme A addresses the determinants of neighborhood opposition to housing development and informs our first research question about which factors should predict opposition. Themes B and C address our second and third research questions by examining the possibilities and limits of applying machine learning to predict political behavior in democratic urban governance.

\subsection{Theme A: Homeownership, Local Interests, and Opposition to Housing}

William Fischel's ``Homevoter Hypothesis'' \citep{fischel2001} provides the foundational theoretical framework for understanding neighborhood opposition to housing development. Fischel argues that homeowners, whose primary asset is typically their residence, rationally oppose development that might reduce property values or alter neighborhood character. This creates systematic political opposition to housing supply increases, even when broader social welfare would benefit. The hypothesis implies that homeownership status, property values, and length of residence should be positively associated with opposition intensity---predictions that our machine learning model will explicitly test using Austin's detailed opposition data.

McCabe's research on homeownership and local political participation extends Fischel's framework by showing that homeowner-dominated municipalities are more likely to restrict multifamily housing and adopt restrictive land use regulations \citep{mccabe2016}. In McCabe's account, the concentration of homeownership within a jurisdiction shapes both policy outputs and the structure of political conflict.

Recent econometric work by Hankinson provides more granular evidence on the determinants of housing opposition using survey experiments across multiple U.S.\ metropolitan areas \citep{hankinson2018}. Hankinson finds that even renters in high-cost markets can exhibit NIMBY attitudes when they experience ``price anxiety'' about future housing costs, particularly when potential new development is sited near their own homes rather than elsewhere in the city.

Drawing on Travis Central Appraisal District records, City Council testimony, and tract-level demographic data, we will test whether the key variables highlighted by Fischel (homeownership status, property value, tenure) and McCabe (homeownership concentration) predict opposition behavior in Austin. If these variables emerge as strong predictors with the expected signs, our results will substantiate these foundational theories in a new empirical context. If other factors dominate---such as proximity to the proposed development, age structure, or past political participation---this would suggest that the existing theories are incomplete or that Austin's particular institutional and demographic context generates distinct dynamics.

\subsection{Theme B: Machine Learning, Political Behavior, and Democratic Legitimacy}

The application of machine learning to predict citizen political behavior raises fundamental questions about democratic governance that have only recently received systematic scholarly attention. Kleinberg et al.\ show that machine learning models can substantially improve prediction of judicial decisions, such as pretrial release and bail, relative to human decision-makers, while emphasizing that such tools can change the decisions they are supposed to predict, raising concerns about feedback effects and the normative acceptability of delegating high-stakes judgments to algorithms \citep{kleinberg2018}. This tension is directly relevant to predicting neighborhood opposition: if city officials used an opposition-forecasting model to pre-emptively manage hearings or outreach, the model could shape the very political behavior it is measuring.

Eubanks' analysis of ``automating inequality'' in welfare, child protective services, and homeless services provides a critical framework for evaluating algorithmic governance tools \citep{eubanks2018}. She documents how predictive systems that are framed as neutral risk-management tools often intensify the surveillance and punishment of poor and marginalized populations and argues that such systems create ``digital poorhouses'' that embed existing inequalities in technical infrastructures. Extending this framework to housing policy, we must ask whether a NIMBY prediction algorithm would systematically identify opposition in particular neighborhoods (for example, wealthier and whiter areas) and whether such identification would advantage development in other areas or stigmatize residents who exercise their democratic right to oppose projects.

Barocas and Selbst provide a widely cited legal-technical analysis of how machine learning can produce discriminatory outcomes even when protected characteristics are not directly used as inputs \citep{barocas2016}. They show how seemingly neutral variables can function as proxies for race, class, or other protected attributes, leading to ``disparate impact'' under antidiscrimination law. In the context of NIMBY prediction, variables such as property values, appraisal protest behavior, or neighborhood characteristics could serve as proxies for race or income, generating predictions that disproportionately affect particular groups even without explicit intent.

The existing literature demonstrates that machine learning can improve prediction in high-stakes settings while raising serious fairness and legitimacy concerns. However, there is little empirical work that builds and evaluates predictive models of citizen political behavior in land use and zoning contexts while simultaneously examining democratic legitimacy concerns. Our study contributes by (1) constructing a parcel- and owner-level predictive model of zoning opposition, and (2) pairing that technical work with qualitative interviews that elicit stakeholders' own evaluations of whether such a tool would be legitimate, fair, or democratically acceptable. This dual approach allows us to assess both the technical feasibility and the democratic acceptability of NIMBY prediction, directly informing our second research question.

\subsection{Theme C: The Limits of Algorithmic Governance in Democratic Contexts}

Beyond fairness concerns, recent scholarship has examined how algorithmic tools reshape the practices and boundaries of democratic governance. Kitchin's critique of ``real-time cities'' argues that smart-city dashboards and data-driven management systems tend to prioritize technocratic efficiency and managerial control over democratic accountability, identifying emerging problems including the politics of urban data, technocratic lock-in, and the risk that optimization logics displace contestation and deliberation \citep{kitchin2014}. These concerns are salient for any algorithmic tool that would allow city officials to anticipate and manage political opposition.

Schuilenburg and Peeters analyze smart-city security systems as expressions of ``pastoral power,'' in which behavioral scripts embedded in urban technologies gently steer citizens toward desired conduct in public space \citep{schuilenburg2018}. Their case study of Eindhoven's ``De-escalate'' project shows how sensor networks and algorithmic interventions can normalize particular uses of space while discouraging others, illustrating how algorithmic systems can subtly reshape who feels welcome to participate in public life and under what conditions.

Zarsky develops an analytic roadmap for evaluating algorithmic decision systems in terms of efficiency, fairness, transparency, and due process \citep{zarsky2016}. He argues that opaque, automated decisions can undermine individuals' ability to understand, contest, and appeal outcomes that affect them, especially when decisions are based on complex statistical models.

Together, this literature suggests that even technically accurate predictive tools can be normatively problematic in democratic governance. Our qualitative interviews with city officials, neighborhood representatives, housing advocates, developers, and civic technologists are designed to probe these concerns directly. We will examine whether stakeholders view NIMBY prediction as a legitimate planning tool, as a form of surveillance that chills democratic participation, or as acceptable only under specific conditions (for example, public transparency about the model and strict limits on how predictions are used). These findings will inform our second and third research questions by identifying the democratic limits of algorithmic governance in housing and by highlighting alternative technological approaches---such as tools for transparency, scenario exploration, or capacity-building---that may improve development processes without profiling or predicting opposition.


\section{Research Design}

This study combines quantitative analysis of existing administrative records with new qualitative interviews and public meeting observations. This design allows us to both identify empirical patterns in opposition behavior and understand how stakeholders evaluate the democratic implications of using predictive tools to anticipate that behavior.

\subsection{Research Site Selection}

Austin, Texas provides an ideal research site for several reasons. First and foremost, Austin is unique among major U.S.\ metropolitan areas in maintaining comprehensive public records of protest petitions and formal opposition to zoning changes. A public information request survey of the top 50 major metros operating under similar state laws revealed that Austin alone has systematically preserved these records in a form amenable to parcel-level analysis. This exceptional data availability makes possible an empirical analysis that would be infeasible in nearly any other U.S.\ city.

Second, the study period spans both zoning stasis and dramatic reform. The concentrated reform period in late 2023, anchored by parking requirement elimination on November 2, 2023 and HOME Phase 1 passage on December 7, 2023, creates a natural experiment in how zoning liberalization affects opposition patterns. These reforms were followed by HOME Phase 2 in May 2024, which further expanded allowable density on corner lots and other properties. This variation allows us to examine whether factors predicting opposition remained stable or shifted as the policy regime changed.

Third, Austin's position as a technology hub with approximately 180,000 tech workers creates both the technical capacity for algorithmic governance and an informed civic discourse about its implications. The city has been recognized nationally for innovative civic technology initiatives, including Open Austin's civic data projects, the city's data-driven approach to hate crime analysis, and digital inclusion programs like Smart Work Learn Play. This context means stakeholders interviewed will likely have more sophisticated understandings of algorithmic tools than in cities with less civic tech infrastructure.

\subsection{Quantitative Component}

\subsubsection*{Data Sources, Linkage, and Management}

The quantitative analysis will construct a parcel-level and owner-level dataset linking multiple administrative and public record sources. Data sources will be obtained and integrated in the following priority order:

\begin{enumerate}
\item \textbf{Opposition records (2007-2025)}: Protest petitions, valid petitions under Texas Local Government Code \S\ 211.006(d) (pre-HB 24), and formal opposition filings extracted from zoning case files via public information requests.
\item \textbf{City of Austin development records (1966-2025, using 2007 onwards)}: Site plan cases and zoning review cases from the City of Austin Open Data Portal, including case outcomes, processing timelines, and parcel identifiers.
\item \textbf{Travis Central Appraisal District records (2021-2025, optionally earlier years via PIR)}: Parcel-level property characteristics including assessed value, land use, and approximate ownership tenure.
\item \textbf{Demographic context (2000-2025)}: U.S.\ Census Bureau tract-level data including decennial census (2000, 2010, 2020) and American Community Survey annual estimates (2005-forward) to characterize neighborhood demographic composition.
\end{enumerate}

These datasets will be linked at the parcel or case level using geocoding and property identifiers. To maintain confidentiality and follow IRB best practices, we will implement a two-table structure:

\begin{itemize}
\item \textbf{Linkage file (Table A)}: Contains study\_id, names, exact addresses, and other direct identifiers. Stored in encrypted, access-controlled files with access limited to the Investigator and authorized research personnel. This file enables auditing of linkages but will not be shared beyond the core research team.
\item \textbf{Analytic dataset (Table B)}: Contains study\_id and derived variables (property value, ownership tenure, distance measures, demographic context, etc.) without direct identifiers. This coded dataset will be used for all analysis and any materials prepared for publication or thesis submission.
\end{itemize}

Direct identifiers will be stored separately from the published analytic dataset, which will use coded study IDs to minimize privacy risks. This approach follows standard coded-data handling for human subjects research and is considered lower risk than maintaining all variables in a single table. For data sources already in the public domain (open data portal, recorded hearings, campaign finance reports), this separation protects against inadvertent re-identification when multiple sources are combined.

Historical data constraints: Opposition records extend back to 2007, providing 18 years of historical data. However, Travis Central Appraisal District parcel characteristics are consistently available only from 2021 forward; earlier years will be requested but may have gaps. For any opposition cases prior to 2021 where appraisal data are incomplete, we will hold property and tax variables fixed at their earliest available values to avoid temporal leakage in backtesting. The annual expanding window validation (described below) will use training cutoffs beginning in 2018, with predictions evaluated in forward time from when the core data sources converge in 2018.

\subsubsection*{Feature Construction and Variables}

The analytic dataset will include three categories of predictors:

\begin{itemize}
\item \textbf{Property and ownership characteristics}: Assessed value, homeownership status, approximate length of ownership, property type, and appraisal protest history.
\item \textbf{Spatial variables}: Proximity to proposed developments, neighborhood characteristics including median income, racial composition, and housing density, and location relative to Austin's 10 council districts and historic neighborhood association boundaries.
\item \textbf{Behavioral and temporal variables}: Prior participation in zoning testimony or protests, campaign contribution patterns, year of development case, and indicators for policy regime changes such as before and after HOME Initiative passage and parking requirement elimination.
\end{itemize}

An additional category will examine demographic differences between property owners and nearby residents, testing whether systematic differences in age, race, or income between the property owner and the surrounding neighborhood are associated with opposition.

\subsubsection*{Modeling Approach and Validation}

The core research question requires predicting which parcels or property owners will participate in formal opposition to specific development cases. This is a binary classification problem well-suited to machine learning approaches. We will implement and compare several model architectures:

\begin{itemize}
\item \textbf{Logistic regression} as an interpretable baseline that provides coefficient estimates for hypothesis testing. This allows us to directly test theoretical predictions from Fischel's homevoter hypothesis, which predicts positive coefficients on property value and ownership tenure, and Hankinson's price anxiety mechanism, which predicts interactions between renter status and neighborhood appreciation rates in high-appreciation areas.
\item \textbf{Tree-based methods}, including Random Forests and Gradient Boosted Trees such as LightGBM, that can capture nonlinear relationships and complex interactions without assuming a specific curve shape in advance.
\end{itemize}

Model performance will be evaluated using classification metrics appropriate for rare events. Because petition-protested cases represent a small fraction of all zoning cases, we will emphasize precision, recall, and F1 score for the positive class, area under the precision-recall curve, calibration plots or Brier scores to assess whether predicted probabilities are well-calibrated, and ROC AUC for comparison to existing literature. We will pay particular attention to recall and precision for the positive class rather than overall accuracy, which can be misleadingly high when the majority class dominates.

To ensure temporal validity and avoid data leakage, we will use an annual expanding window validation strategy rather than a single train-test split. This mimics how the model would be deployed in practice, with annual retraining on accumulating historical data and forward prediction. The structure allows us to assess whether opposition patterns identified in earlier years remain predictive after the 2023 reforms.

Beyond predictive accuracy, the analysis will examine which variables emerge as most important in the models. Variable importance metrics, such as permutation importance for tree-based models or coefficient magnitudes for logistic regression, will indicate whether homeownership, property values, and spatial proximity dominate predictions as theory suggests, or whether other factors like prior participation history or demographic differences are more influential. This directly addresses our first research question about which factors predict opposition.

\subsection{Qualitative Component}

The qualitative component will involve approximately 10 semi-structured interviews with stakeholders who directly participate in or are affected by Austin's housing development processes. We will use purposive sampling to recruit a small but diverse set of participants from each stakeholder category, targeting roughly 10 total interviews to target at least two distinct perspectives in each category. We do not claim to reach thematic saturation; the qualitative component is designed to surface major tensions and use-cases rather than exhaust the space of views.

The targeted sample includes: city officials or planners, neighborhood association leaders, housing advocates such as Austinites for Urban Rail Action (AURA) members, civic technologists with expertise in civic data or algorithmic governance in Austin, and (optionally) residential developers who have navigated opposition to their projects.

Participants will be recruited through the Investigator's existing professional networks in Austin planning and housing policy. Initial contacts will be made via email or person-to-person invitation describing the study's purpose, voluntary nature, and expected time commitment. Interviews will be semi-structured, following a prepared interview guide while allowing flexibility to explore emerging themes. Interviews will last approximately 15--30 minutes and may be conducted in person, by phone, or via secure videoconference depending on participant preference. With participant consent, interviews will be audio-recorded for transcription and analysis. Participants will have the option to participate without recording, in which case detailed notes will be taken instead.

In addition to interviews, the study will include non-participant observation of public meetings over a six-month period. Observation will include both live video attendance at City Council and Planning \& Zoning Commission meetings and review of archived ATXN recordings. Field notes will document how stakeholders discuss predictive tools, data-driven governance, or concerns about surveillance and democratic participation in the development context. Observations will focus on publicly observable behavior and statements already part of the public record.

Interview recordings will be professionally transcribed, with transcripts checked for accuracy. Analysis will follow an iterative thematic coding approach using qualitative analysis software such as NVivo or Atlas.ti. Deductive codes will be adapted from established frameworks in the algorithmic governance literature, particularly Grimmelikhuijsen and Meijer's legitimacy framework (input, throughput, and output legitimacy) and procedural justice concepts such as fairness, transparency, accountability, and concerns about chilling effects on participation. Inductive codes will emerge from close reading of early transcripts, capturing themes and concerns that do not fit neatly into existing frameworks.

The analysis will identify dominant themes in how different stakeholder groups perceive predictive tools, points of consensus and divergence across stakeholder categories, conditions and concerns that shape whether tools are seen as legitimate, alternative technological approaches that stakeholders propose, and connections between stakeholder positions and their structural interests. Qualitative data will also help distinguish between concerns that stem from misunderstandings of what predictive models do and concerns that identify genuine risks or governance pitfalls.

\subsection{Integration of Quantitative and Qualitative Findings}

The mixed-methods design allows us to triangulate findings and address different aspects of our overarching research question. The quantitative component establishes what patterns exist in opposition behavior and which factors predict them. The qualitative component examines how stakeholders interpret the benefits, risks, and legitimacy of using those patterns predictively in planning practice.

Integration will occur at the interpretation stage. The primary novelty of this project is not in discovering entirely new concepts of fairness or legitimacy in algorithmic governance---these are well-established concerns in the literature. Rather, the novelty lies in applying these frameworks to a specific, high-stakes, and previously understudied domain: predictive modeling of NIMBY opposition at the parcel level in housing and land use planning. By pairing detailed longitudinal data on protest petitions with focused qualitative investigation of stakeholder reactions, we provide empirical grounding for debates about algorithmic governance that have often remained theoretical or focused on other domains like criminal justice or welfare administration.


\section{Task Schedule}
{\small

\noindent\textbf{I.\quad 2025 October--November: Literature Review and Data Acquisition}
\begin{itemize}[noitemsep,leftmargin=2em]
    \item 10/01--11/15: Complete comprehensive literature review
    \item 11/25: Submit IRB materials (target approval mid-December)
    \item 10/15-12/15: Acquire datasets via public information requests
    \item \textbf{Milestone}: Complete thesis proposal (12/10)
\end{itemize}

\noindent\textbf{II.\quad 2025 December -- 2026 January: Quantitative Analysis}
\begin{itemize}[noitemsep,leftmargin=2em]
    \item 12/15--01/15: Data preparation, cleaning, and linkage; construct two-table structure
    \item 01/10--01/25: Feature construction and exploratory analysis
    \item 01/15--02/10: ML model development with annual expanding window validation
    \item 01/20--02/10: Geographic and temporal analysis of opposition patterns
    \item \textbf{Deliverable}: Draft quantitative methods and results chapter (02/10)
\end{itemize}

\noindent\textbf{III.\quad 2026 January--February: Qualitative Data Collection}
\begin{itemize}[noitemsep,leftmargin=2em]
    \item 01/05--01/25: Finalize interview protocol, recruit participants via purposive sampling
    \item 01/20--02/28: Conduct 10 interviews (15--30 minutes each)
    \item 01/15--03/15: Non-participant observation (live attendance and archived ATXN review)
    \item 02/01--03/10: Transcribe and begin preliminary coding
    \item \textbf{Deliverable}: Complete interview transcripts and field notes (03/10)
    \item \textbf{Note}: Schedule in-person Austin time during this period, secure Columbia travel funding
\end{itemize}

\noindent\textbf{IV.\quad 2026 February--March: Integration and Analysis}
\begin{itemize}[noitemsep,leftmargin=2em]
    \item 02/20-03/20: Systematic qualitative coding
    \item 03/01--03/15: Analyze comparative frameworks from other cities
    \item 03/10--03/25: Integrate quantitative and qualitative findings
    \item 03/15--03/30: Develop theoretical framework connecting findings to HB 24 context
    \item \textbf{Deliverable}: Complete draft chapters submitted to advisor (03/30)
\end{itemize}

\noindent\textbf{V.\quad 2026 March--May: Writing and Defense}
\begin{itemize}[noitemsep,leftmargin=2em]
    \item 04/01--04/10: Revise based on advisor feedback
    \item 04/03: Distribute penultimate draft to reader(s) (14 days before jury, required)
    \item 04/05--04/15: Prepare jury presentation
    \item 04/13--04/17: Thesis jury (required for graduation)
    \item 04/18--05/05: Final revisions incorporating jury feedback
    \item \textbf{Final submission}: 05/08/2026
\end{itemize}
}

% References
\section*{References}
\addcontentsline{toc}{section}{References}

\begin{thebibliography}{99}

\bibitem[Barocas \& Selbst(2016)]{barocas2016}
Barocas, S., \& Selbst, A. D. (2016). Big data's disparate impact. \textit{California Law Review}, 104(3), 671--732.

\bibitem[City of Austin(2010)]{cityaustin2010}
City of Austin. (2010). \textit{City of Austin Charter}. Austin, TX.

\bibitem[Eubanks(2018)]{eubanks2018}
Eubanks, V. (2018). \textit{Automating inequality: How high-tech tools profile, police, and punish the poor}. St. Martin's Press.

\bibitem[Fischel(2001)]{fischel2001}
Fischel, W. A. (2001). \textit{The homevoter hypothesis: How home values influence local government taxation, school finance, and land-use policies}. Harvard University Press.

\bibitem[Hankinson(2018)]{hankinson2018}
Hankinson, M. (2018). When do renters behave like homeowners? High rent, price anxiety, and NIMBYism. \textit{American Political Science Review}, 112(3), 473--493.

\bibitem[IDC(2024)]{idc2024}
IDC Government Insights. (2024, February 26). Finalists named in IDC Government Insights' seventh annual Smart Cities North America Awards. \url{https://my.idc.com/getdoc.jsp?containerId=prUS51902824}

\bibitem[Kitchin(2014)]{kitchin2014}
Kitchin, R. (2014). The real-time city? Big data and smart urbanism. \textit{GeoJournal}, 79(1), 1--14.

\bibitem[Kleinberg et al.(2018)]{kleinberg2018}
Kleinberg, J., Lakkaraju, H., Leskovec, J., Ludwig, J., \& Mullainathan, S. (2018). Human decisions and machine predictions. \textit{The Quarterly Journal of Economics}, 133(1), 237--293.

\bibitem[KVUE(2024)]{kvue2024vacancy}
KVUE. (2024). Austin apartment vacancy rates climb to 10-15\%. KVUE News.

\bibitem[McCabe(2016)]{mccabe2016}
McCabe, B. J. (2016). \textit{No place like home: Wealth, community, and the politics of homeownership}. Oxford University Press.

\bibitem[Open Austin(2025)]{efa2025}
Electronic Frontier Alliance. (2025, August 30). Open Austin: Reimagining civic engagement and digital equity in Texas. Electronic Frontier Foundation.

\bibitem[Pool(2023)]{pool2023}
Pool, K. (2023). HOME Initiative Phase 1. Austin City Council.

\bibitem[Schuilenburg \& Peeters(2018)]{schuilenburg2018}
Schuilenburg, M., \& Peeters, R. (2018). Smart cities and the architecture of security: Pastoral power and the scripted design of public space. \textit{City, Territory and Architecture}, 5(1), 1--12.

\bibitem[Travis CAD(n.d.)]{tcad_public}
Travis Central Appraisal District. (n.d.). Public information. \url{https://traviscad.org/publicinformation}

\bibitem[Travis County Clerk(2022)]{tcclerk2022}
Travis County Clerk. (2022). November 2022 election results. Travis County, TX.

\bibitem[Unlock MLS(2025)]{unlockMLS2025}
Unlock MLS. (2025). Austin housing market report, August 2025.

\bibitem[US Census Bureau(2020)]{census2020}
U.S. Census Bureau. (2020). 2020 Decennial Census.

\bibitem[Wells(2023)]{axios2023parking}
Wells, N. (2023, November 2). Austin eliminates parking requirements citywide. \textit{Axios Austin}.

\bibitem[Zarsky(2016)]{zarsky2016}
Zarsky, T. (2016). The trouble with algorithmic decisions: An analytic road map to examine efficiency and fairness in automated and opaque decision making. \textit{Science, Technology, \& Human Values}, 41(1), 118--132.

\end{thebibliography}

\appendix

\section{Interview Guide}

\textbf{Predicting NIMBYism: Stakeholder Interview Guide}

\textbf{Introduction}
\begin{itemize}[leftmargin=2em]
    \item Thank you for agreeing to speak with me. I am a graduate student at Columbia University studying housing development processes in Austin.
    \item This interview will last approximately 15-30 minutes.
    \item Participation is voluntary. You may decline to answer any question or stop at any time.
\end{itemize}

\textit{If recording}: Do I have your permission to record this conversation for accuracy?

\textbf{I. Professional Background and Context}
\begin{enumerate}[leftmargin=2em]
    \item Could you briefly describe your role and how you interact with housing development or zoning processes in Austin?
    \item Based on your experience, what are the most significant factors that currently delay or prevent housing development in Austin?
    \item How would you characterize the nature of community engagement or neighborhood opposition to revisions/re-zonings in recent years?
\end{enumerate}

\textbf{II. Perceptions of Predictive Tools}
\begin{enumerate}
    \item Are you aware of any data-driven or algorithmic tools currently used by the City of Austin (e.g., in permitting, service delivery, or planning) that you find particularly useful or problematic?
    \item Cities are increasingly using data to model urban trends. What is your initial reaction to the concept of using data to predict where neighborhood opposition to zoning changes is most likely to occur?
    \item What potential benefits, if any, could you see from using such a tool? (For example: better resource allocation, earlier outreach, identifying engagement gaps).
    \item What risks or concerns would you have about such a tool? (For example: bias, transparency, fairness, exclusion).
\end{enumerate}

\textbf{III. Democratic Legitimacy and Governance}
\begin{enumerate}
    \item Do you believe that using predictive analytics to anticipate public participation would make the planning process more or less democratic? Why?
    \item If the City were to adopt more predictive technologies for planning, what specific safeguards or transparency measures would be essential for you to trust them?
    \item Is there anything else about the intersection of technology, data, and housing policy in Austin that you would like to share?
\end{enumerate}

\section{Recruitment Materials}

\textbf{Email Subject:} Interview Request: Research on Austin Housing Policy and Technology

Dear [Name],

I am writing to invite you to participate in a research study regarding housing development processes in Austin. I am a graduate student in Urban Planning at Columbia University, conducting this research for a thesis under the supervision of Dr. Hiba Bou Akar.

\textbf{Purpose of the Study:}
This study examines patterns of neighborhood opposition to housing development and evaluates how different stakeholders perceive the use of new data-driven and predictive tools in planning. We are seeking perspectives from [City Officials / Neighborhood Leaders / Housing Advocates / Developers / Civic Technologists] to understand diverse views on transparency, efficiency, and fairness in the development process.

\textbf{Procedures:}
Participation involves a single semi-structured interview lasting approximately 15-30 minutes. The interview can be conducted in person, by phone, or via secure videoconference (e.g., Zoom).

\textbf{Voluntary Participation:}
Your participation is completely voluntary. You may decline to answer any questions or withdraw from the study at any time without penalty.

If you are willing to share your perspective, please let me know your availability for a brief conversation in the coming weeks.

Thank you for your time and consideration.

Sincerely,

Daniel Lewis, Investigator\\
Graduate School of Architecture, Planning and Preservation\\
Columbia University\\
dl3645@columbia.edu

\section{Public Meeting Observation Protocol}

\textbf{Date}: \_\_\_\_\_\_\_\_\_\_\_\_\_ \\
\textbf{Location}: \_\_\_\_\_\_\_\_\_\_\_\_\_ \\
\textbf{Meeting Type}: $\Box$ City Council \hspace{0.5cm} $\Box$ Planning \& Zoning Commission \hspace{0.5cm} $\Box$ Neighborhood Association \\
\textbf{Format}: $\Box$ Live attendance \hspace{0.5cm} $\Box$ Archived ATXN video review

\textbf{Attendees} (general categories, no names):
\begin{itemize}[leftmargin=2em]
    \item Approximate number present: \_\_\_\_\_\_\_
    \item Stakeholder groups represented: \_\_\_\_\_\_\_
\end{itemize}

\textbf{Topics Discussed Related to Study}:
\begin{itemize}[leftmargin=2em]
    \item Housing development or zoning matters: \_\_\_\_\_\_\_
    \item References to data, predictive tools, or algorithmic approaches: \_\_\_\_\_\_\_
    \item Discussion of neighborhood opposition or participation: \_\_\_\_\_\_\_
\end{itemize}

\textbf{Field Notes}:
[Space for detailed notes on how stakeholders discuss housing development processes, predictive tools, democratic participation, and related themes. Focus on publicly observable statements and behavior that are already part of the public record.]

\textbf{Follow-up Questions or Observations}:

\end{document}

