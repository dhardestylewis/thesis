\documentclass[12pt]{article}

% Packages
\usepackage[margin=1in]{geometry}
\usepackage{times}
\usepackage{setspace}
\usepackage{natbib}
\usepackage{graphicx}
\usepackage{booktabs}
\usepackage{longtable}
\usepackage{array}
\usepackage{hyperref}
\usepackage[table]{xcolor}

% Document settings
\doublespacing
\setlength{\parindent}{0.5in}
\bibliographystyle{chicago}

% Title and author
\title{Austin's Housing Policy Transformation: A Fact-Checked Analysis (2020-2025)}
\author{[Author Name]\\Columbia University\\School of International and Public Affairs}
\date{January 2025}

\begin{document}

\maketitle

\begin{abstract}
This study examines Austin, Texas's transformation in housing policy between 2020 and 2025, focusing on verified policy changes, electoral shifts, and market outcomes. Through careful fact-checking of claims and sources, this analysis documents how Austin implemented significant zoning reforms including the HOME Initiative, parking mandate elimination, and density bonus programs. The city reduced minimum lot sizes from 5,750 to 2,500 square feet, eliminated parking requirements citywide in November 2023, and saw permits increase from approximately 15,000 in 2023 to 28,000 in 2024. While the reforms achieved measurable increases in housing production and contributed to rent declines, they also coincided with ongoing displacement of communities of color and faced implementation challenges including Development Services Department budget cuts of 24 positions in FY 2024-25. This fact-checked analysis provides a reliable account of Austin's housing transformation based solely on verifiable sources and confirmed data.
\end{abstract}

\newpage
\tableofcontents
\newpage

\section{Introduction}

On December 7, 2023, Austin's city council approved the first phase of the HOME (Housing Opportunities for Middle-Income Earners) Initiative by a 9-2 vote, allowing up to three units on single-family lots and reducing minimum lot sizes to 2,500 square feet.\footnote{Austin Monitor, "Council Approves HOME Phase 1," December 7, 2023; City of Austin Ordinance 20231207-001.} This vote marked a significant shift in Austin's approach to housing policy after the failure of the CodeNEXT land development code rewrite.

The CodeNEXT effort, which began in 2012 and cost over \$10 million, was effectively killed by a March 18, 2020 ruling from Judge Jan Soifer that found the city had violated Texas Local Government Code notice requirements.\footnote{KUT, "Judge Throws Out City Council Votes On New Austin Land Code," March 18, 2020.} The 14th Court of Appeals in Houston (via transfer from the Third Court) upheld this ruling on March 17, 2022, ending the decade-long effort.\footnote{KUT, "Appeals court confirms Austin's land code rewrite skirted state law," March 17, 2022.}

This study examines Austin's subsequent housing policy transformation through 2025, relying exclusively on verified sources and confirmed data.

\section{Political Context and Elections}

\subsection{The 2022 Elections}

The November 2022 elections resulted in significant changes to Austin's city council composition. Kirk Watson won the mayoral race, defeating Celia Israel in a runoff by just 17 votes (50.01\% to 49.99\%).\footnote{Travis County Clerk, "December 2022 Runoff Election Results," December 13, 2022.} This election, combined with district races, created what observers described as a pro-housing majority on the council.

\subsection{The 2024 Elections}

In November 2024, several key council races occurred:
\begin{itemize}
\item District 6: Krista Laine defeated incumbent Mackenzie Kelly by approximately 704 votes\footnote{The Texas Tribune, "Austin Election Results," November 2024.}
\item District 7: Mike Siegel won in a December 14, 2024 runoff by approximately 206 votes\footnote{Austin Monitor, December 15, 2024.}
\item District 10: Marc Duchen won the open seat\footnote{AURA, "Election Results Analysis," November 2024.}
\end{itemize}

\section{Major Housing Reforms}

\subsection{HOME Initiative}

The HOME Initiative was developed in 2023 under the leadership of Council Member Chito Vela and others. Council Member Leslie Pool, who had previously opposed density increases, co-sponsored the initiative in July 2023, marking what local media described as a significant shift in her position.\footnote{Austin Monitor, "Pool Authors HOME Initiative," July 2023.}

\subsubsection{Phase 1 (December 7, 2023)}
HOME Phase 1 passed 9-2 with the following provisions:\footnote{City of Austin Ordinance 20231207-001, December 7, 2023.}
\begin{itemize}
\item Allows up to 3 units on all residential lots
\item Reduces minimum lot size from 5,750 to 2,500 square feet
\item Reduces front setbacks to 15 feet
\item Removes adult occupancy limits
\item Defines "Tiny Home" as 400 square feet or less
\end{itemize}

Through October 7, 2025, the city reported:
\begin{itemize}
\item 578 applications submitted
\item 453 applications approved
\item 903 housing units accommodated\footnote{City of Austin Development Services Department, "HOME Phase 1 Update," October 7, 2025.}
\end{itemize}

\subsubsection{Phase 2 (May 16, 2024)}
HOME Phase 2 passed 9-2 with provisions including:
\begin{itemize}
\item Allows up to 10 units on corner lots
\item Further reduces setback requirements
\item Modifies compatibility standards
\end{itemize}

\subsection{Parking Mandate Elimination}

Austin eliminated citywide parking requirements in November 2023, becoming the largest U.S. city to do so after San José (which eliminated requirements in December 2022).\footnote{Austin Texas, "Parking Requirements Eliminated," November 2023; The Texas Tribune, November 2023.} The measure passed 10-1 with only Council Member Mackenzie Kelly opposing.

Analysis of 14 Affordability Unlocked projects showed developers built approximately 25\% fewer parking spaces than would have been required under old rules.\footnote{KUT, "Parking Analysis," September 2024.}

\subsection{Density Bonus Programs}

\subsubsection{DB90 Program}
On February 29, 2024, the council approved the DB90 program by a 9-1-1 vote (Alison Alter against, Mackenzie Kelly abstaining).\footnote{Austin Monitor, February 29, 2024.} The program offers:
\begin{itemize}
\item Up to 90\% additional floor area
\item In exchange for 10\% affordable units at 50\% MFI (rental) or 12\% at 60\% MFI (rental) or 12\% at 80\% MFI (ownership)
\item 30-foot height bonus up to 90 feet total
\end{itemize}

\subsubsection{Program Performance Data}
According to Rachel Tepper, Principal Planner, in a November 2023 presentation to the Housing and Planning Committee:\footnote{Austin Monitor, "Density Bonus Performance Report," November 2023.}
\begin{itemize}
\item Affordability Unlocked: 5,343 units planned, 481 completed (9.0\% completion rate)
\item University Neighborhood Overlay: 572 units completed
\item Downtown Density Bonus: 111 units planned, 81 completed, \$6.5M fees received
\item VMU Programs: 1,535 units planned, 755 completed (49.2\% completion rate)
\item Total programs operating: 13-15 (varying by reporting period)
\end{itemize}

\subsection{Compatibility Standards Reform}

On May 16, 2024, the council modified compatibility standards with a 10-1 vote.\footnote{City of Austin Ordinance 20240516-004, effective July 15, 2024.} The reforms:
\begin{itemize}
\item Apply within 75 feet of a "triggering property"
\item Set caps of 40 feet within 50 feet and 60 feet within 75 feet
\item Became effective July 15, 2024
\end{itemize}

\subsection{ETOD Overlay}

The Equitable Transit-Oriented Development overlay was adopted May 16, 2024, effective July 15, 2024:\footnote{City of Austin, "ETOD Policy," May 16, 2024.}
\begin{itemize}
\item Allows 120 feet of height within 1/4 mile of transit stations
\item Allows 90 feet between 1/4 and 1/2 mile from stations
\item Applies to 28 designated transit station areas
\end{itemize}

\section{Market Outcomes}

\subsection{Housing Production}

Building permits increased substantially:
\begin{itemize}
\item 2023: Approximately 15,000 permits
\item 2024: Approximately 28,000 permits (86\% increase)\footnote{City of Austin Development Services Department, 2024.}
\item September 2024 surge: 314 permits for 2-4 unit developments versus 30 in August and 25 in July\footnote{Team Price Real Estate using City of Austin data, September 2024.}
\end{itemize}

As of early 2025, approximately 21,000 units were under construction for delivery within the year.\footnote{Newmark/CoStar data via multiple local sources, 2025.}

\subsection{Rent Trends}

Austin experienced significant rent changes:
\begin{itemize}
\item Peak (August 2022): \$1,726 for 2-bedroom average\footnote{RealPage, CoStar, August 2022.}
\item April 2025: \$1,431 for 2-bedroom average (17\% decline from peak)\footnote{Team Price Real Estate, April 2025.}
\item Austin had among the largest rent declines of major U.S. metros in 2024\footnote{Redfin, "Rent Report," 2024.}
\end{itemize}

\subsection{Home Prices}

Median home prices (Metro Statistical Area):
\begin{itemize}
\item January 2020: Approximately \$308,000\footnote{Team Price Realty, January 2020.}
\item May 2022 peak: \$550,000\footnote{Austin Board of Realtors, May 2022.}
\item January 2025: \$409,765\footnote{Unlock MLS, January 2025.}
\item August 2025: \$444,490\footnote{Unlock MLS, August 2025.}
\end{itemize}

August 2025 market statistics:
\begin{itemize}
\item Months of inventory: 5.9
\item Active listings: 14,220
\item Days on market: 68\footnote{All from Unlock MLS, August 2025.}
\end{itemize}

\subsection{Vacancy Rates}

Apartment vacancy rates increased from approximately 4\% in late 2021 to 10-15\% by mid-2025, with variation by source and submarket.\footnote{RealPage, CoStar, various reports 2021-2025.}

\section{Implementation Challenges}

\subsection{Development Services Department}

Contrary to adding staff, DSD actually faced significant cuts:
\begin{itemize}
\item FY 2024-25: Cut approximately 24 full-time positions
\item FY 2025-26: Additional 55 positions planned for cuts
\item Attributed to "contraction in development activity"\footnote{Community Impact, July 2024.}
\end{itemize}

However, process improvements were achieved:
\begin{itemize}
\item Phase 1 improvements (October 2023-May 2024) reduced initial review times from 87-99 days to 32 days (56\% improvement)
\item Follow-up cycles reduced from 50 to approximately 15 days\footnote{Austin Monitor, October 2024; DSD Director José G. Roig memo.}
\end{itemize}

\subsection{Infrastructure Capacity}

Austin Water faced documented capacity challenges:
\begin{itemize}
\item Walnut Creek Wastewater Treatment Plant exceeded 90\% capacity for three months in 2024
\item City Council approved \$1 billion expansion in May 2024
\item Construction planned 2025-2031 to add 25 million gallons/day capacity\footnote{Austin Water, Austin Monitor, March 2025.}
\end{itemize}

Austin Energy experienced transformer shortages in 2023 due to supply chain issues, with deliveries dropping 90\% in Q3 2022 compared to Q1.\footnote{Austin Energy press release, May 2023.}

\section{State Legislative Context}

\subsection{HB 2127 ("Death Star" Bill)}

House Bill 2127, passed in 2023, prohibits local governments from adopting ordinances exceeding state standards in multiple regulatory areas. After initial legal challenges, the Third Court of Appeals upheld the law on July 18, 2025.\footnote{Third Court of Appeals ruling, July 18, 2025; KUT coverage.}

\subsection{Housing-Related Legislation (89th Session, 2025)}

\begin{itemize}
\item \textbf{SB 840}: Requires cities to allow multifamily housing on commercially zoned land (effective September 1, 2025)\footnote{Texas Legislature, SB 840, 2025.}
\item \textbf{HB 24}: Reforms valid petition process, removing 3/4 supermajority requirement (effective September 1, 2025)\footnote{Texas Legislature, HB 24, 2025.}
\end{itemize}

\section{Demographic Patterns}

\subsection{Population Changes by Race/Ethnicity}

According to City Demographer Lila Valencia's July 2025 presentation:\footnote{Austin Monitor, July 2025.}
\begin{itemize}
\item Austin is one of only seven U.S. cities where white population growth accounted for 100\% or more of total growth from 2010-2020
\item Black population decreased by approximately 5,000 residents from 2010-2020
\item Hispanic population grew modestly in absolute numbers but declined as a percentage (35.1\% to 32.5\%)
\item Growth in communities of color occurred primarily in suburban counties
\end{itemize}

\subsection{Essential Worker Residency}

Selena Xie, President of Austin EMS Association, stated in November 2024 that approximately 70\% of Austin EMS Association members live outside city limits.\footnote{Austin Monitor, November 2024.} (Note: The document claimed 75\%, which overstates by 5 percentage points.)

\section{Key Stakeholders and Organizations}

\subsection{AURA (Austinites for Urban Rail Action)}

AURA evolved from a transit advocacy group founded in 2014 to Austin's primary YIMBY organization, growing to over 2,000 members by 2022.\footnote{AURA, "Membership Report," 2022.} The organization played a significant role in mobilizing support for housing reforms.

\subsection{YIMBYtown Conference}

Austin hosted YIMBYtown 2024 on February 26-28, 2024, with approximately 500 attendees. Seven Austin City Council members plus Mayor Kirk Watson participated.\footnote{Austin Chronicle, March 1, 2024.} (Note: Only one conference was held in Austin, not two as sometimes claimed.)

\section{City Leadership}

T.C. Broadnax was appointed City Manager on April 4, 2024 (unanimous council vote) and took office May 6, 2024, with a base salary of \$470,017.60.\footnote{City of Austin press release, April 4, 2024; KUT.} He replaced Jesús Garza, who served as Interim City Manager after Spencer Cronk was fired February 15, 2023.

\section{Conclusion}

Austin's housing policy transformation between 2020 and 2025 involved significant regulatory changes that can be verified through public records. The HOME Initiative, parking mandate elimination, and density bonus programs represent substantial shifts in city policy. Housing production increased measurably, with permits rising from approximately 15,000 to 28,000 between 2023 and 2024. Rents declined from peak levels, though affordability challenges persist.

However, implementation faced challenges including Development Services Department staff cuts rather than additions, with 24 positions eliminated in FY 2024-25. Infrastructure capacity constraints at Austin Water required a \$1 billion expansion approval. Demographic patterns showed continued displacement of communities of color despite increased housing production.

This fact-checked analysis, based solely on verifiable sources, provides a reliable but necessarily limited account of Austin's housing transformation. Many claims that appear in other narratives about this period cannot be independently verified and have been excluded from this analysis.

\appendix
\section{Verified Data Tables}

\begin{table}[h]
\centering
\caption{Major Housing Reforms (Verified Votes and Dates)}
\begin{tabular}{llll}
\toprule
Date & Reform & Vote & Source \\
\midrule
Nov 2023 & Parking Elimination & 10-1 & Austin Texas, Texas Tribune \\
Dec 7, 2023 & HOME Phase 1 & 9-2 & Ordinance 20231207-001 \\
Feb 29, 2024 & DB90 & 9-1-1 & Austin Monitor \\
May 16, 2024 & HOME Phase 2 & 9-2 & City records \\
May 16, 2024 & Compatibility Reform & 10-1 & Ordinance 20240516-004 \\
May 16, 2024 & ETOD Overlay & 10-1 & City of Austin \\
\bottomrule
\end{tabular}
\end{table}

\begin{table}[h]
\centering
\caption{Housing Market Metrics (Verified Data Only)}
\begin{tabular}{lll}
\toprule
Metric & Value & Source \\
\midrule
Median Home Price (Jan 2025) & \$409,765 & Unlock MLS \\
Median Home Price (Aug 2025) & \$444,490 & Unlock MLS \\
2-BR Rent (Peak 2022) & \$1,726 & RealPage \\
2-BR Rent (April 2025) & \$1,431 & Team Price \\
Permits 2023 & ~15,000 & City of Austin \\
Permits 2024 & ~28,000 & City of Austin \\
Vacancy Rate 2025 & 10-15\% & Multiple sources \\
\bottomrule
\end{tabular}
\end{table}

\begin{table}[h]
\centering
\caption{Density Bonus Program Performance (November 2023)}
\begin{tabular}{lrr}
\toprule
Program & Units Planned & Units Completed \\
\midrule
Affordability Unlocked & 5,343 & 481 (9\%) \\
University Neighborhood & --- & 572 \\
Downtown Density Bonus & 111 & 81 \\
VMU Programs & 1,535 & 755 (49\%) \\
\bottomrule
\end{tabular}
\small{Source: Rachel Tepper presentation to Housing Committee, November 2023}
\end{table}

\bibliography{references}

\end{document}