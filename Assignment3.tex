\documentclass[12pt]{article}

% Packages
\usepackage[margin=1in]{geometry}
\usepackage{times}
\usepackage{setspace}
\usepackage{booktabs}
\usepackage{longtable}
\usepackage{array}
\usepackage{hyperref}
\usepackage{enumitem}
\usepackage[style=apa,backend=biber]{biblatex}

% Bibliography configuration
\addbibresource{references.bib}

% Document settings
\doublespacing
\setlength{\parindent}{0.5in}

% Title and author
\title{Assignment 3: Understanding NIMBY Opposition Patterns in Austin's Housing Reform Era}
\author{Daniel Lewis\\
dl3645@columbia.edu\\
Columbia University\\
Urban Planning}
\date{\today}

\begin{document}

\maketitle

\section{Introduction: Austin's Housing Crisis and Political Transformation}

Austin, Texas faces an unprecedented housing affordability crisis that has fundamentally reshaped its political landscape. Between 2010 and 2020, Austin's population grew by 33\%, adding nearly 250,000 residents to reach 978,908 \autocite{census2020}. This growth, driven primarily by the technology sector expansion, created severe housing supply constraints. By May 2022, the median home price reached \$550,000, up 79\% from January 2020's \$308,000. As of August 2025, prices have moderated to \$444,490 but remain unaffordable for most residents \autocite{unlockMLS2025}.

The crisis has displaced long-term residents, particularly communities of color. Austin's Black population declined by approximately 5,000 residents from 2010-2020, with the city's Black population share falling to 7.8\%. The Hispanic population, while growing modestly in absolute numbers, saw its share of the city population decline from 35.1\% to 32.5\% as gentrification pushed residents to suburban areas \autocite{census2020}. An October 2023 memo from the Austin EMS employee association reported that nearly 78\% of medics who work for the city and county live outside Austin, reflecting the growing disconnect between public-safety wages and housing costs in the city.\footnote{Austin EMS Association memo to City Council, October 24, 2023. Available at \url{https://services.austintexas.gov/edims/document.cfm?id=418778}}

Austin's governance structure complicates housing policy implementation. The city operates under a council-manager system with 10 single-member districts plus an at-large mayor. The mayor has minimal executive authority---one vote equal to other council members, no veto power, and no ability to hire department heads. All executive functions rest with the City Manager, who reports to all 11 council members equally. This structure means building council supermajorities is essential for passing controversial reforms, as Texas law requires three-quarters votes to overcome property owner protests against zoning changes.\footnote{Tex. Loc. Gov't Code § 211.006(d).}

In response to the crisis, Austin underwent a dramatic political transformation. The November 2022 elections delivered a pro-housing supermajority, with candidates supporting increased housing production winning all five contested seats \autocite{tcclerk2022,watson2022campaign}. Most dramatically, Council Member Leslie Pool, who had campaigned on ``Protect Single-Family Zoning'' platforms,\footnote{The Austin Independent, December 19, 2023, reporting on Pool's 2016 campaign materials.} reversed her position in July 2023 to author the HOME (Housing Opportunities for Middle-Income Earners) Initiative \autocite{pool2023}. This political shift enabled sweeping reforms: eliminating parking requirements citywide (November 2023) \autocite{axios2023parking,strongtowns2023parking}, reducing minimum lot sizes from 5,750 to 1,800 square feet \autocite{kut2023home}, and allowing up to 10 units on corner lots \autocite{austinmonitor2024home2}.

The reforms have generated measurable market impacts. Austin's apartment vacancy rate climbed from about 4\% in late 2021\footnote{Based on CoStar data reported in market analyses. Team Price Real Estate reports 3.96\% in September 2021.} to approximately 10-15\% by mid-2025 \autocite{kvue2024vacancy}. However, permit data from the City of Austin shows complex trends, with total building permits fluctuating from 15,270 in 2022 to 12,023 in 2023 and 11,576 in 2024,\footnote{City of Austin Open Data Portal, Annual Number of Building Permits Issued dataset.} suggesting implementation challenges despite policy reforms. The Development Services Department planned to eliminate about two dozen positions in the FY 2024-25 budget despite increased application volume.\footnote{Community Impact, July 30, 2024. Available at \url{https://communityimpact.com/austin/south-central-austin/government/2024/07/30/austin-weighs-development-staffing-cuts-as-local-construction-slows/}} Site plan review times were reduced from roughly 100 days to about 32 days after process improvements,\footnote{Ibid.} though new delays have emerged from the surge in complex applications.

Throughout this transformation, neighborhood associations have mounted organized opposition to development projects, using tools like valid petitions, public testimony, and legal challenges. Understanding the patterns and predictors of this opposition has become crucial for implementing housing reforms effectively. Austin's unique position as a technology hub---home to Dell, Oracle, Tesla, Apple, and roughly 180,000 tech workers \autocite{axios2024techjobs}---provides both the technical capability and the civic context to examine whether data-driven approaches could improve housing policy implementation.

\section{Research Questions}

This project aims to answer the following \textbf{motivating question}:

\textbf{How can cities identify and understand patterns of neighborhood opposition to housing development, and what are the democratic implications of using predictive analytics to anticipate such opposition?}

Sub-questions:

\begin{enumerate}
\item[a.] What demographic, geographic, and behavioral factors predict whether residents will oppose specific development projects in Austin?

\item[b.] How do different stakeholder groups (homeowners, renters, housing advocates, city officials) perceive the legitimacy and utility of algorithmic tools in housing policy implementation?

\item[c.] What alternative technological approaches could improve housing development processes without raising concerns about democratic participation?
\end{enumerate}

\section{Describe the ``so what''}

This research addresses a critical gap at the intersection of urban planning, democratic governance, and algorithmic decision-making. As cities nationwide struggle with housing affordability, many are turning to technological solutions to streamline development approval. However, the use of predictive algorithms to identify potential opposition raises fundamental questions about democratic participation and civil liberties that have not been systematically examined.

The findings will directly inform policy decisions in Austin and other cities considering similar tools. Austin's Development Services Department and Planning Commission could use insights about opposition patterns to improve community engagement and resource allocation. Housing advocates could better understand how to build coalitions and address legitimate concerns. Most importantly, the research will establish ethical frameworks for cities considering algorithmic tools in democratic processes.

This work contributes to broader literature on algorithmic governance, smart cities, and housing policy. It provides empirical evidence about the tradeoffs between efficiency and legitimacy in urban governance, informing theoretical debates about technocratic versus democratic approaches to city management.

\newpage

\section{Research Design Grid}

\begin{longtable}{p{4cm}|p{5.5cm}|p{5.5cm}}
\toprule
\textbf{Research Question} & \textbf{Data Sources} & \textbf{Details/Analysis} \\
\midrule
\endfirsthead

\multicolumn{3}{c}{\textit{(continued from previous page)}} \\
\toprule
\textbf{Research Question} & \textbf{Data Sources} & \textbf{Details/Analysis} \\
\midrule
\endhead

\midrule
\multicolumn{3}{r}{\textit{(continued on next page)}} \\
\endfoot

\bottomrule
\endlastfoot

\multicolumn{3}{l}{\textbf{MAJOR RESEARCH QUESTION 1}} \\*
\midrule

What demographic, geographic, and behavioral factors predict whether residents will oppose specific development projects in Austin? &
\begin{itemize}[noitemsep,topsep=0pt,parsep=0pt,partopsep=0pt,leftmargin=*]
\item \href{https://traviscad.org/publicinformation/}{Travis Central Appraisal District} parcel-level appraisal and protest records (2018-2025), obtained via public information requests
\item \href{https://www.austintexas.gov/department/city-council/council/council_meeting_info_center.htm}{City Council meeting testimony transcripts} and \href{https://www.austintexas.gov/page/atxn-video-archive}{ATXN video recordings} (2018-2025) --- Methodology: Manual reading of selected transcripts, skimming of large sample, automated analysis of remainder
\item Campaign contribution records (\href{https://www.ethics.state.tx.us}{Texas Ethics Commission})
\item Census demographic data by tract (\href{https://www.census.gov}{U.S. Census Bureau})
\item Development application data from \href{https://data.austintexas.gov/Building-and-Development/Site-Plan-Cases/mavg-96ck}{Site Plan Cases} and \href{https://data.austintexas.gov/dataset/Zoning-Review-Cases_data/qwte-z96m}{Zoning Review Cases} (City of Austin Open Data Portal)
\end{itemize} &
\begin{itemize}[noitemsep,topsep=0pt,parsep=0pt,partopsep=0pt,leftmargin=*]
\item Machine learning model to identify predictive factors
\item Variables: proximity to development, home value, length of residence, age, previous testimony
\item Validation on 2018-2022 cases, testing on 2023-2025
\item Geographic analysis of opposition clustering
\item Temporal analysis of opposition evolution
\end{itemize} \\

\midrule

\newpage

\multicolumn{3}{l}{\textbf{MAJOR RESEARCH QUESTION 2}} \\*
\midrule

How do different stakeholder groups (homeowners, renters, housing advocates, city officials) perceive the legitimacy and utility of algorithmic tools in housing policy implementation? &
\begin{itemize}[noitemsep,topsep=0pt,parsep=0pt,partopsep=0pt,leftmargin=*]
\item Semi-structured interviews (n=15):
\begin{itemize}[noitemsep,topsep=0pt]
\item 3 city officials/planners
\item 3 neighborhood association leaders
\item 3 housing advocates (AURA members)
\item 3 developers
\item 3 civic technologists
\end{itemize}
\item Observations at City Council, Planning \& Zoning Commission, and Neighborhood Association meetings (6 months)
\item Document analysis of public comments
\end{itemize} &
\begin{itemize}[noitemsep,topsep=0pt,parsep=0pt,partopsep=0pt,leftmargin=*]
\item Thematic analysis of interview transcripts
\item Stakeholder mapping of concerns and benefits
\item Comparison of perceived vs actual algorithmic capabilities
\item Analysis of democratic legitimacy concerns
\item Trust and transparency themes
\end{itemize} \\

\midrule

\multicolumn{3}{l}{\textbf{MINOR RESEARCH QUESTION 3}} \\*
\midrule

What alternative technological approaches could improve housing development processes without raising concerns about democratic participation? &
\begin{itemize}[noitemsep,topsep=0pt,parsep=0pt,partopsep=0pt,leftmargin=*]
\item Frameworks to reference from other cities:
\begin{itemize}[noitemsep,topsep=0pt]
\item Seattle traffic signal optimization (Project Green Light with Google AI)
\item Boston Residential Displacement Risk Map (city risk-scoring tool)
\item San Francisco sentiment mapping research (Twitter/311 analysis prototypes)
\end{itemize}
\item Tools to reference:
\begin{itemize}[noitemsep,topsep=0pt]
\item \href{https://www.austintexas.gov/abc-applications}{Austin Build + Connect (AB+C) Portal} usage metrics
\item City of Austin Residential Stormwater Management Program Single-Family Development Calculator usage data
\item \href{https://housingworksaustin.org/}{City of Austin Housing \& Planning Department dashboards} and Strategic Housing Blueprint analytics
\end{itemize}
\end{itemize} &
\begin{itemize}[noitemsep,topsep=0pt,parsep=0pt,partopsep=0pt,leftmargin=*]
\item Review of different cities' approaches to inform our own analysis structure
\item Characterization of tool types and their democratic implications
\item Best practices identification for Austin context
\item Framework development for ethical deployment
\end{itemize} \\

\end{longtable}

\section{Literature Review}

This section organizes the literature around three themes that correspond to the three research questions introduced above. Theme A addresses the determinants of neighborhood opposition to housing development and informs our first research question about which factors should predict opposition. Themes B and C address our second and third research questions by examining the possibilities and limits of applying machine learning to predict political behavior in democratic urban governance.

\subsection{Theme A: Homeownership, Local Interests, and Opposition to Housing}

William Fischel's ``Homevoter Hypothesis'' \autocite{fischel2001} provides the foundational theoretical framework for understanding neighborhood opposition to housing development. Fischel argues that homeowners, whose primary asset is typically their residence, rationally oppose development that might reduce property values or alter neighborhood character. This creates systematic political opposition to housing supply increases, even when broader social welfare would benefit. The hypothesis implies that homeownership status, property values, and length of residence should be positively associated with opposition intensity---predictions that our machine learning model will explicitly test using Austin's detailed opposition data.

McCabe's research on homeownership and local political participation extends Fischel's framework by showing that homeowner-dominated municipalities are more likely to restrict multifamily housing and adopt restrictive land use regulations \autocite{mccabe2016}. In McCabe's account, the concentration of homeownership within a jurisdiction shapes both policy outputs and the structure of political conflict. Our analysis will test whether variation in homeownership concentration across Austin neighborhoods is associated with the patterns of zoning opposition that McCabe's theory would predict.

Recent econometric work by Hankinson provides more granular evidence on the determinants of housing opposition using survey experiments across multiple U.S.\ metropolitan areas \autocite{hankinson2018}. Hankinson finds that even renters in high-cost markets can exhibit NIMBY attitudes when they experience ``price anxiety'' about future housing costs, particularly when potential new development is sited near their own homes rather than elsewhere in the city. Our machine learning approach can test whether Hankinson's price-anxiety mechanism operates in Austin by examining whether opposition patterns differ between homeowners and long-term renters in rapidly appreciating neighborhoods. If our predictive model finds that renter opposition increases in high-appreciation areas, it would provide novel support for Hankinson's theory using revealed-preference behavioral data rather than survey responses.

\textbf{How our research tests these theories.} Drawing on Travis Central Appraisal District records, City Council testimony, and tract-level demographic data, we will test whether the key variables highlighted by Fischel (homeownership status, property value, tenure) and McCabe (homeownership concentration) predict opposition behavior in Austin. If these variables emerge as strong predictors with the expected signs, our results will substantiate these foundational theories in a new empirical context. If other factors dominate---such as proximity to the proposed development, age structure, or past political participation---this would suggest that the existing theories are incomplete or that Austin's particular institutional and demographic context generates distinct dynamics.

\subsection{Theme B: Machine Learning, Political Behavior, and Democratic Legitimacy}

The application of machine learning to predict citizen political behavior raises fundamental questions about democratic governance that have only recently received systematic scholarly attention. Kleinberg et al.\ show that machine learning models can substantially improve prediction of judicial decisions, such as pretrial release and bail, relative to human decision-makers \autocite{kleinberg2018}. At the same time, they emphasize that such tools can change the decisions they are supposed to predict, raising concerns about feedback effects and the normative acceptability of delegating high-stakes judgments to algorithms. This tension is directly relevant to predicting neighborhood opposition: if city officials used an opposition-forecasting model to pre-emptively manage hearings or outreach, the model could shape the very political behavior it is measuring.

Eubanks' analysis of ``automating inequality'' in welfare, child protective services, and homeless services provides a critical framework for evaluating algorithmic governance tools \autocite{eubanks2018}. She documents how predictive systems that are framed as neutral risk-management tools often intensify the surveillance and punishment of poor and marginalized populations. Eubanks argues that such systems create ``digital poorhouses'' that embed existing inequalities in technical infrastructures. Extending this framework to housing policy, we must ask whether a NIMBY prediction algorithm would systematically identify opposition in particular neighborhoods (for example, wealthier and whiter areas) and whether such identification would advantage development in other areas or stigmatize residents who exercise their democratic right to oppose projects.

Barocas and Selbst provide a widely cited legal-technical analysis of how machine learning can produce discriminatory outcomes even when protected characteristics are not directly used as inputs \autocite{barocas2016}. They show how seemingly neutral variables can function as proxies for race, class, or other protected attributes, leading to ``disparate impact'' under antidiscrimination law. In the context of NIMBY prediction, variables such as property values, appraisal protest behavior, or neighborhood characteristics could serve as proxies for race or income, generating predictions that disproportionately affect particular groups even without explicit intent.

\textbf{How our research advances this literature.} The existing literature demonstrates that machine learning can improve prediction in high-stakes settings while raising serious fairness and legitimacy concerns. However, there is little empirical work that builds and evaluates predictive models of citizen political behavior in land use and zoning contexts while simultaneously examining democratic legitimacy concerns. Our study contributes by (1) constructing a parcel- and owner-level predictive model of zoning opposition, and (2) pairing that technical work with qualitative interviews that elicit stakeholders' own evaluations of whether such a tool would be legitimate, fair, or democratically acceptable. This dual approach allows us to assess both the technical feasibility and the democratic acceptability of NIMBY prediction, directly informing our second research question.

\subsection{Theme C: The Limits of Algorithmic Governance in Democratic Contexts}

Beyond fairness concerns, recent scholarship has examined how algorithmic tools reshape the practices and boundaries of democratic governance. Kitchin's critique of ``real-time cities'' argues that smart-city dashboards and data-driven management systems tend to prioritize technocratic efficiency and managerial control over democratic accountability \autocite{kitchin2014}. He identifies emerging problems including the politics of urban data, technocratic lock-in, and the risk that optimization logics displace contestation and deliberation. These concerns are salient for any algorithmic tool that would allow city officials to anticipate and manage political opposition.

Schuilenburg and Peeters analyze smart-city security systems as expressions of ``pastoral power,'' in which behavioral scripts embedded in urban technologies gently steer citizens toward desired conduct in public space \autocite{schuilenburg2018}. Their case study of Eindhoven's ``De-escalate'' project shows how sensor networks and algorithmic interventions can normalize particular uses of space while discouraging others. Although their focus is on security and public order, the broader lesson is that algorithmic systems can subtly reshape who feels welcome to participate in public life and under what conditions.

Zarsky develops an analytic roadmap for evaluating algorithmic decision systems in terms of efficiency, fairness, transparency, and due process \autocite{zarsky2016}. He argues that opaque, automated decisions can undermine individuals' ability to understand, contest, and appeal outcomes that affect them, especially when decisions are based on complex statistical models. In the housing opposition context, a predictive model that profiles certain owners or neighborhoods as likely opponents could influence how officials treat their testimony or whether they prioritize outreach to particular groups, raising questions about procedural fairness and the right to contest decisions.

\textbf{How our research contributes.} Together, this literature suggests that even technically accurate predictive tools can be normatively problematic in democratic governance. Our qualitative interviews with city officials, neighborhood representatives, housing advocates, developers, and civic technologists are designed to probe these concerns directly. We will examine whether stakeholders view NIMBY prediction as a legitimate planning tool, as a form of surveillance that chills democratic participation, or as acceptable only under specific conditions (for example, public transparency about the model and strict limits on how predictions are used). These findings will inform our second and third research questions by identifying the democratic limits of algorithmic governance in housing and by highlighting alternative technological approaches---such as tools for transparency, scenario exploration, or capacity-building---that may improve development processes without profiling or predicting opposition.

\section{Timeline}

This research will proceed through the following phases:

\subsection{Phase I: Literature Review and Data Acquisition (October--November 2025)}

\begin{itemize}
\item Complete comprehensive literature review
\item Obtain IRB approval for human subjects research
\item Acquire datasets via public information requests and open data portals
\end{itemize}

\subsection{Phase II: Quantitative Analysis (December 2025--January 2026)}

\begin{itemize}
\item Data preparation and cleaning
\item ML model development and validation (2018-2022 training, 2023-2025 testing)
\item Geographic and temporal analysis of opposition patterns
\end{itemize}

\subsection{Phase III: Qualitative Data Collection (January--February 2026)}

\begin{itemize}
\item Conduct interviews (n=15)
\item Observe public meetings (City Council, Planning \& Zoning, Neighborhood Associations)
\item Collect and analyze documentary evidence
\item Transcribe and begin coding interview data
\end{itemize}

\subsection{Phase IV: Integration and Analysis (February--March 2026)}

\begin{itemize}
\item Integrate quantitative and qualitative findings
\item Analyze Austin's deployed tools and comparative frameworks from other cities
\item Develop theoretical framework connecting findings to broader questions of algorithmic governance and democratic legitimacy
\end{itemize}

\subsection{Phase V: Writing and Defense (March--May 2026)}

\begin{itemize}
\item Draft complete thesis manuscript
\item Revise based on advisor feedback
\item Prepare and deliver defense presentation
\item Final revisions and submission (May 2026)
\end{itemize}

% Print bibliography
\printbibliography[title=References]

\end{document}
